% Options for packages loaded elsewhere
\PassOptionsToPackage{unicode}{hyperref}
\PassOptionsToPackage{hyphens}{url}
\PassOptionsToPackage{dvipsnames,svgnames,x11names}{xcolor}
%
\documentclass[
  latterpaper,
  DIV=11,
  numbers=noendperiod]{scrartcl}

\usepackage{amsmath,amssymb}
\usepackage{iftex}
\ifPDFTeX
  \usepackage[T1]{fontenc}
  \usepackage[utf8]{inputenc}
  \usepackage{textcomp} % provide euro and other symbols
\else % if luatex or xetex
  \usepackage{unicode-math}
  \defaultfontfeatures{Scale=MatchLowercase}
  \defaultfontfeatures[\rmfamily]{Ligatures=TeX,Scale=1}
\fi
\usepackage{lmodern}
\ifPDFTeX\else  
    % xetex/luatex font selection
\fi
% Use upquote if available, for straight quotes in verbatim environments
\IfFileExists{upquote.sty}{\usepackage{upquote}}{}
\IfFileExists{microtype.sty}{% use microtype if available
  \usepackage[]{microtype}
  \UseMicrotypeSet[protrusion]{basicmath} % disable protrusion for tt fonts
}{}
\makeatletter
\@ifundefined{KOMAClassName}{% if non-KOMA class
  \IfFileExists{parskip.sty}{%
    \usepackage{parskip}
  }{% else
    \setlength{\parindent}{0pt}
    \setlength{\parskip}{6pt plus 2pt minus 1pt}}
}{% if KOMA class
  \KOMAoptions{parskip=half}}
\makeatother
\usepackage{xcolor}
\setlength{\emergencystretch}{3em} % prevent overfull lines
\setcounter{secnumdepth}{5}
% Make \paragraph and \subparagraph free-standing
\ifx\paragraph\undefined\else
  \let\oldparagraph\paragraph
  \renewcommand{\paragraph}[1]{\oldparagraph{#1}\mbox{}}
\fi
\ifx\subparagraph\undefined\else
  \let\oldsubparagraph\subparagraph
  \renewcommand{\subparagraph}[1]{\oldsubparagraph{#1}\mbox{}}
\fi


\providecommand{\tightlist}{%
  \setlength{\itemsep}{0pt}\setlength{\parskip}{0pt}}\usepackage{longtable,booktabs,array}
\usepackage{calc} % for calculating minipage widths
% Correct order of tables after \paragraph or \subparagraph
\usepackage{etoolbox}
\makeatletter
\patchcmd\longtable{\par}{\if@noskipsec\mbox{}\fi\par}{}{}
\makeatother
% Allow footnotes in longtable head/foot
\IfFileExists{footnotehyper.sty}{\usepackage{footnotehyper}}{\usepackage{footnote}}
\makesavenoteenv{longtable}
\usepackage{graphicx}
\makeatletter
\def\maxwidth{\ifdim\Gin@nat@width>\linewidth\linewidth\else\Gin@nat@width\fi}
\def\maxheight{\ifdim\Gin@nat@height>\textheight\textheight\else\Gin@nat@height\fi}
\makeatother
% Scale images if necessary, so that they will not overflow the page
% margins by default, and it is still possible to overwrite the defaults
% using explicit options in \includegraphics[width, height, ...]{}
\setkeys{Gin}{width=\maxwidth,height=\maxheight,keepaspectratio}
% Set default figure placement to htbp
\makeatletter
\def\fps@figure{htbp}
\makeatother
% definitions for citeproc citations
\NewDocumentCommand\citeproctext{}{}
\NewDocumentCommand\citeproc{mm}{%
  \begingroup\def\citeproctext{#2}\cite{#1}\endgroup}
\makeatletter
 % allow citations to break across lines
 \let\@cite@ofmt\@firstofone
 % avoid brackets around text for \cite:
 \def\@biblabel#1{}
 \def\@cite#1#2{{#1\if@tempswa , #2\fi}}
\makeatother
\newlength{\cslhangindent}
\setlength{\cslhangindent}{1.5em}
\newlength{\csllabelwidth}
\setlength{\csllabelwidth}{3em}
\newenvironment{CSLReferences}[2] % #1 hanging-indent, #2 entry-spacing
 {\begin{list}{}{%
  \setlength{\itemindent}{0pt}
  \setlength{\leftmargin}{0pt}
  \setlength{\parsep}{0pt}
  % turn on hanging indent if param 1 is 1
  \ifodd #1
   \setlength{\leftmargin}{\cslhangindent}
   \setlength{\itemindent}{-1\cslhangindent}
  \fi
  % set entry spacing
  \setlength{\itemsep}{#2\baselineskip}}}
 {\end{list}}
\usepackage{calc}
\newcommand{\CSLBlock}[1]{\hfill\break\parbox[t]{\linewidth}{\strut\ignorespaces#1\strut}}
\newcommand{\CSLLeftMargin}[1]{\parbox[t]{\csllabelwidth}{\strut#1\strut}}
\newcommand{\CSLRightInline}[1]{\parbox[t]{\linewidth - \csllabelwidth}{\strut#1\strut}}
\newcommand{\CSLIndent}[1]{\hspace{\cslhangindent}#1}

\KOMAoption{captions}{tableheading}
\makeatletter
\@ifpackageloaded{tcolorbox}{}{\usepackage[skins,breakable]{tcolorbox}}
\@ifpackageloaded{fontawesome5}{}{\usepackage{fontawesome5}}
\definecolor{quarto-callout-color}{HTML}{909090}
\definecolor{quarto-callout-note-color}{HTML}{0758E5}
\definecolor{quarto-callout-important-color}{HTML}{CC1914}
\definecolor{quarto-callout-warning-color}{HTML}{EB9113}
\definecolor{quarto-callout-tip-color}{HTML}{00A047}
\definecolor{quarto-callout-caution-color}{HTML}{FC5300}
\definecolor{quarto-callout-color-frame}{HTML}{acacac}
\definecolor{quarto-callout-note-color-frame}{HTML}{4582ec}
\definecolor{quarto-callout-important-color-frame}{HTML}{d9534f}
\definecolor{quarto-callout-warning-color-frame}{HTML}{f0ad4e}
\definecolor{quarto-callout-tip-color-frame}{HTML}{02b875}
\definecolor{quarto-callout-caution-color-frame}{HTML}{fd7e14}
\makeatother
\makeatletter
\@ifpackageloaded{caption}{}{\usepackage{caption}}
\AtBeginDocument{%
\ifdefined\contentsname
  \renewcommand*\contentsname{Tabla de contenidos}
\else
  \newcommand\contentsname{Tabla de contenidos}
\fi
\ifdefined\listfigurename
  \renewcommand*\listfigurename{Listado de Figuras}
\else
  \newcommand\listfigurename{Listado de Figuras}
\fi
\ifdefined\listtablename
  \renewcommand*\listtablename{Lista de Cuadros}
\else
  \newcommand\listtablename{Lista de Cuadros}
\fi
\ifdefined\figurename
  \renewcommand*\figurename{Figura}
\else
  \newcommand\figurename{Figura}
\fi
\ifdefined\tablename
  \renewcommand*\tablename{Cuadro}
\else
  \newcommand\tablename{Cuadro}
\fi
}
\@ifpackageloaded{float}{}{\usepackage{float}}
\floatstyle{ruled}
\@ifundefined{c@chapter}{\newfloat{codelisting}{h}{lop}}{\newfloat{codelisting}{h}{lop}[chapter]}
\floatname{codelisting}{Listado}
\newcommand*\listoflistings{\listof{codelisting}{Listado de Listados}}
\makeatother
\makeatletter
\makeatother
\makeatletter
\@ifpackageloaded{caption}{}{\usepackage{caption}}
\@ifpackageloaded{subcaption}{}{\usepackage{subcaption}}
\makeatother
\ifLuaTeX
\usepackage[bidi=basic]{babel}
\else
\usepackage[bidi=default]{babel}
\fi
\babelprovide[main,import]{spanish}
% get rid of language-specific shorthands (see #6817):
\let\LanguageShortHands\languageshorthands
\def\languageshorthands#1{}
\ifLuaTeX
  \usepackage{selnolig}  % disable illegal ligatures
\fi
\usepackage{bookmark}

\IfFileExists{xurl.sty}{\usepackage{xurl}}{} % add URL line breaks if available
\urlstyle{same} % disable monospaced font for URLs
\hypersetup{
  pdftitle={Post With Code},
  pdfauthor={Harlow Malloc},
  pdflang={es},
  colorlinks=true,
  linkcolor={blue},
  filecolor={Maroon},
  citecolor={Blue},
  urlcolor={Blue},
  pdfcreator={LaTeX via pandoc}}

\title{Post With Code}
\author{Harlow Malloc}
\date{2024-07-03}

\begin{document}
\maketitle

\renewcommand*\contentsname{Tabla de contenidos}
{
\hypersetup{linkcolor=}
\setcounter{tocdepth}{3}
\tableofcontents
}
La ciencia y el poder han mantenido una relación compleja a lo largo de
la historia. Mientras que la ciencia busca el conocimiento y la verdad,
el poder, en muchas ocasiones, intenta moldear los resultados
científicos para servir a sus propios intereses. Esta tensión se
manifiesta en varios problemas contemporáneos que enfrenta la ciencia.
Según Infobae (2016), entre estos problemas destacan siete en
particular: la crisis financiera, el mal diseño de muchos estudios, la
escasa revalidación de investigaciones, el fracaso del sistema de
revisión por pares, el alto costo de acceso a artículos científicos, la
transmisión errónea de la ciencia y el estrés significativo en la vida
de los jóvenes científicos.

\begin{tcolorbox}[enhanced jigsaw, colframe=quarto-callout-note-color-frame, toprule=.15mm, colbacktitle=quarto-callout-note-color!10!white, coltitle=black, bottomtitle=1mm, opacityback=0, toptitle=1mm, rightrule=.15mm, arc=.35mm, title=\textcolor{quarto-callout-note-color}{\faInfo}\hspace{0.5em}{Nota}, titlerule=0mm, breakable, colback=white, leftrule=.75mm, bottomrule=.15mm, left=2mm, opacitybacktitle=0.6]

La ciencia y el poder han mantenido una relación compleja a lo largo de
la historia. Mientras que la ciencia busca el conocimiento y la verdad,
el poder, en muchas ocasiones, intenta moldear los resultados
científicos para servir a sus propios intereses. Esta tensión se
manifiesta en varios problemas contemporáneos que enfrenta la ciencia.
Según Infobae (2016), entre estos problemas destacan siete en
particular: la crisis financiera, el mal diseño de muchos estudios, la
escasa revalidación de investigaciones, el fracaso del sistema de
revisión por pares, el alto costo de acceso a artículos científicos, la
transmisión errónea de la ciencia y el estrés significativo en la vida
de los jóvenes científicos.

\end{tcolorbox}

\begin{tcolorbox}[enhanced jigsaw, colframe=quarto-callout-warning-color-frame, toprule=.15mm, colbacktitle=quarto-callout-warning-color!10!white, coltitle=black, bottomtitle=1mm, opacityback=0, toptitle=1mm, rightrule=.15mm, arc=.35mm, title=\textcolor{quarto-callout-warning-color}{\faExclamationTriangle}\hspace{0.5em}{Advertencia}, titlerule=0mm, breakable, colback=white, leftrule=.75mm, bottomrule=.15mm, left=2mm, opacitybacktitle=0.6]

Estos desafíos requieren una reconfiguración profunda de cómo se
practica la ciencia. Un cambio de paradigma hacia una ciencia más
orientada a soluciones reales y accesibles es crucial (Velázquez-Pérez y
Velázquez-Velázquez, 2023). En este contexto, la ciencia abierta se
presenta como una solución prometedora. Esta aboga por la transparencia,
accesibilidad y colaboración en la investigación científica.

\end{tcolorbox}

\begin{tcolorbox}[enhanced jigsaw, colframe=quarto-callout-important-color-frame, toprule=.15mm, colbacktitle=quarto-callout-important-color!10!white, coltitle=black, bottomtitle=1mm, opacityback=0, toptitle=1mm, rightrule=.15mm, arc=.35mm, title=\textcolor{quarto-callout-important-color}{\faExclamation}\hspace{0.5em}{Importante}, titlerule=0mm, breakable, colback=white, leftrule=.75mm, bottomrule=.15mm, left=2mm, opacitybacktitle=0.6]

Según Anglada y Abadal (2018), el proyecto Foster, uno de los portales
más completos sobre ciencia abierta, visualiza este concepto como un
panal de abejas, donde cada celda representa un componente esencial.
Entre estos componentes se encuentran los cuadernos abiertos (open
notebooks), datos abiertos, revisión abierta, acceso abierto (open
access), software libre, redes sociales académicas, ciencia ciudadana y
recursos educativos abiertos.

Es importante diferenciar entre ciencia abierta y acceso abierto.
Mientras que el acceso abierto se refiere principalmente a la
disponibilidad gratuita de artículos científicos acceso abierto, la
ciencia abierta engloba un espectro más amplio de prácticas, incluyendo
la apertura de datos, métodos y la participación pública en la
investigación ciencia abierta (Mejía-Ponce, 2023).

\end{tcolorbox}

\begin{tcolorbox}[enhanced jigsaw, colframe=quarto-callout-tip-color-frame, toprule=.15mm, colbacktitle=quarto-callout-tip-color!10!white, coltitle=black, bottomtitle=1mm, opacityback=0, toptitle=1mm, rightrule=.15mm, arc=.35mm, title=\textcolor{quarto-callout-tip-color}{\faLightbulb}\hspace{0.5em}{Tip}, titlerule=0mm, breakable, colback=white, leftrule=.75mm, bottomrule=.15mm, left=2mm, opacitybacktitle=0.6]

Las motivaciones para adoptar la ciencia abierta son variadas: desde el
deseo de mejorar la reproducibilidad y transparencia, hasta la necesidad
de democratizar el acceso al conocimiento.

\end{tcolorbox}

\begin{tcolorbox}[enhanced jigsaw, colframe=quarto-callout-caution-color-frame, toprule=.15mm, colbacktitle=quarto-callout-caution-color!10!white, coltitle=black, bottomtitle=1mm, opacityback=0, toptitle=1mm, rightrule=.15mm, arc=.35mm, title=\textcolor{quarto-callout-caution-color}{\faFire}\hspace{0.5em}{Precaución}, titlerule=0mm, breakable, colback=white, leftrule=.75mm, bottomrule=.15mm, left=2mm, opacitybacktitle=0.6]

El establecimiento de un marco para la ciencia abierta enfrenta varios
desafíos, como la resistencia al cambio por parte de algunas
instituciones, la falta de infraestructura adecuada y la necesidad de
cambiar las normas de evaluación científica (García-Peñalvo, 2017;
Fressoli y Arza, 2018). Además, la ética y la calidad de la información
son pilares fundamentales en este modelo, por lo que publicar en el
ecosistema de la ciencia abierta requiere adherirse a estándares
rigurosos que aseguren la integridad y precisión de los datos (Antunes
et al., 2020).

\end{tcolorbox}

\section{Conclusiones}

En Latinoamérica, la implementación de normativas en este nuevo contexto
presenta tendencias y desafíos específicos. México, por ejemplo, ha
experimentado una crisis en su sistema científico debido a la Ley
Federal de Austeridad Republicana de Estado y la emisión del memorándum
regulador de medidas de austeridad, este último documento ``regula'' el
uso del presupuesto en ciencia y tecnología. La crisis de la ciencia
mexicana y los esfuerzos significativos por hacer ciencia abierta en el
país ejemplifican los desafíos y oportunidades que se enfrentan en la
región (Vázquez-Reyes, 2019; García-Guerrero, 2024; Peña, 2024). En
conclusión, la ciencia abierta representa una vía hacia una práctica
científica más ética, accesible y colaborativa, pero su implementación
requiere un compromiso concertado y una visión estratégica para superar
las barreras actuales.

\section{Bibliografía}

Abadal, E. (2021). Ciencia abierta: un modelo con piezas por encajar.
\emph{Arbor, 197}(799), a588-a588.

Anglada, L., y Abadal, E. (2018). ¿Qué es la ciencia abierta?
\emph{Anuario ThinkEPI, 12}, 292-298.

Antunes, M. D. L., Sanches, T., Lopes, C., y Alonso-Arévalo, J. (2020).
Publicar en el ecosistema de la ciencia abierta. \emph{Cuadernos de
Documentación Multimedia, (31)}, 1-11.

\phantomsection\label{refs}
\begin{CSLReferences}{1}{0}
\bibitem[\citeproctext]{ref-mejuxeda-ponce2023}
Mejía-Ponce, Ma. E. (2023). Ciencia abierta y acceso abierto: una
aproximación conceptual a la ética y la calidad de la información.
\emph{Dixi}, \emph{26}(1), 1-28.
\url{https://doi.org/10.16925/2357-5891.2024.01.04}

\end{CSLReferences}



\end{document}
